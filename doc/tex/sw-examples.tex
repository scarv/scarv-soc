
% list each example
% - name, purpose, source files, how to run.
%

The project includes example programs.
These demonstrate SoC functionality, how to use it, and how to
build programs that use the \SCARVSOC BSP.
Shared code used by multiple example programs is kept in
\SOCDIR{src/examples/share}.
The Makefile which discovers and builds example programs is
\SOCDIR{src/examples/Makefile.in}.
Studying this will explain how the build flow for example programs
works.

\subsubsection{UART Example}

The UART example program shows how to use the UART peripheral in
the BSP, and acts as an ``echo" program: echoing any characters the
user types into their terminal back to them.

\begin{itemize}

\item Source files for the UART example are kepts in
    \SOCDIR{src/examples/uart}.

\item The UART example program can be built with:

\begin{lstlisting}[language=bash,style=block]
> make example-uart
\end{lstlisting}

    This will compile the BSP (if needed), and link it against the
    complied example program.
    It will also build a verilated model which is setup to load the
    relevent memory images.
    Build artifacts are placed in
    \SOCDIR[\SOCWORK]{examples/uart}.

\item The UART example program can be run using:

\begin{lstlisting}[language=bash,style=block]
> make run-example-uart
cp \$SOC_WORK/fsbl/fsbl.hex  \$SOC_WORKexamples/uart/rom.hex
cp \$SOC_WORK/verilator/verilated    \$SOC_WORK/examples/uart/verilated
cd \$SOC_WORK/examples/uart ; ./verilated +TIMEOUT=10000000 +WAIT
>> ./verilated +TIMEOUT=10000000 +WAIT
>> Timeout after 10000000 cycles.
>> Open Slave UART at: /dev/pts/5
>> Press return to begin simulation:
\end{lstlisting}

    This will start the verilated model and open an emulated UART
    port, which you can connect your favourite serial terminal application
    too.

    After connecting your serial terminal to the correct port (you do
    not need to set the correct BAUD rate for the emulated port), return
    to the terminal session where the verilated model is running and
    press return.

    You can now start typing in the serial terminal.
    There should be a welcome message, followed by an echo of whatever
    you type.
    Typing ``!" aborts the UART program.

\end{itemize}

Note that Section \ref{sec:hw:fpga:upload} describes how to
upload and run the example UART program on the FPGA, rather than
in sumulation.

