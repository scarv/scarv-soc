
The \SCARVSOC implements the following functionality:

\begin{itemize}

\item Designed for enabling experimentation.
    
    \begin{itemize}
    \item Capable of running completely inside a simulation environment.

    \item Implementation primarily targets FPGAs.

    \item The number of FPGA specific blocks within the design is
        minimised.  Some vendor specific IP blocks may be used for
        bring-up, but these will be be replaced with free/open-source
        components eventually.

    \item Implementable on an ASIC with minimal changes.
    \end{itemize}

\item Uses the \SCARVCPU \footnote{\url{https://github.com/scarv/scarv-cpu}}
      core.

    \begin{itemize}
    \item 5-stage pipelined RISC-V {\tt RV32I} implementation.
    \item {\bf M}ultiply standard extension support.
    \item {\bf C}ompressed standard extension support.
    \item M-mode privilidged resource architecture implemented.
    \item Precise exceptions.
    \item Instruction retired and cycle performance counters.
    \item External, software and timer interrupts.
    \item Implements the XCrypto
        \footnote{\url{https://github.com/scarv/xcrypto}}
        instruction set extensions for secure and efficient Cryptography.
    \item Optimised for cryptographic side-channel security.
    \item Capable of running the Zephyr
        \footnote{\url{https://zephyrproject.org}}
        RTOS.
    \end{itemize}

\item 1K ROM containing a customisable first stage bootloader.

\item 64Kb of tightly coupled SRAM or BRAM.

\item Support for multiple peripherals:

    \begin{itemize}
    \item UART Serial Interface.
    \item 16 GPIO Pins.
    \item One external interrupt pin.
    \end{itemize}

\end{itemize}

