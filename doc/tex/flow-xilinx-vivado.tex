
Here we describe the flow for re-creating the Xilinx project used to generate
a bitstream for the complete SoC.
The files for this flow live in {\tt \$SOC\_HOME/flow/xilinx}:

\begin{itemize}
\item {\tt scarv-soc-project.tcl} - The Vivado `tcl` script which creates 
    the project.

\item {\tt tb\_top\_behave.wcfg} - Simulation wave view configuration.

\item {\tt constraints-xc7k.xdc} - Pin constraints for targeting the
    SASEBO-GIII FPGA board.

\item {\tt scarv-soc-xc7k.bit} - A pre-generated bitfile.
\end{itemize}

\noindent Other relevent files are:

\begin{itemize}
\item {\tt bin/create-xilinx-xc7k-project.sh} - A bash script to automate the
    project re-creation steps.

\item {\tt rtl/xilinx/*} - Xilinx project specific RTL files.

\item {\tt verif/xilinx/*} - Testbench files for running simulations inside
Vivado.
\end{itemize}


\noindent Run the following commands to re-create the project:

\begin{itemize}

\item Make sure that the SOC environment is setup:

\begin{lstlisting}[style=block, language=bash]
> source ./bin/conf.sh
\end{lstlisting}

\item Make sure that the {\tt VIVADO\_ROOT} variable points to a Vivado
      installation:

\begin{lstlisting}[style=block, language=bash]
> export VIVADO\_ROOT=<path to installation>
\end{lstlisting}

    Note that the project was initially created using Vivado 2019.1.

\item Source the Vivado environment setup scripts:

\begin{lstlisting}[style=block, language=bash]
> source \$VIVADO\_ROOT/settings64.sh
\end{lstlisting}

\item Run the project re-creation scripts:

\begin{lstlisting}[style=block, language=bash]
> cd \$SOC\_HOME
> ./bin/create-xilinx-xc7k-project.sh
\end{lstlisting}

\item You can now open Vivado and the project file in
  {\tt \$SOC\_WORK/vivado/scarv-soc-xc7k/scarv-soc-xc7k.xpr}
  and start working with the project.

  There is also a pre-generated bitfile in 
  {\tt \$SOC\_HOME/flow/xilinx/}.

  Note that running this script will force compilation of
  the first stage boot loader (FSBL) described in 
  Section \ref{sec:sw:fsbl}.
  This is required because the FSBL is automatically pre-loaded
  into the ROM by Vivado during bitstream creation, so this
  file must exist.


\end{itemize}


\subsubsection{Uploading programs to the FPGA}
\label{sec:hw:fpga:upload}

Once the FPGA is programmed with the bitstream, and assuming that
the FSBL has been correctly integrated into the ROM, programs can
be uploaded to the FPGA using the flow described here.
It is recommended to familiarise oneself with how the FSBL operates
(Section \ref{sec:sw:fsbl}) and the associated
code (\SOCDIR{src/fsbl}) before continuing.

\begin{enumerate}
\item Ensure that the FPGA has been programmed with the
      generated bitstream using Vivado.

\item Open a terminal, and open the UART port used to communicate with
      the FPGA. If you are using Ubuntu, and a simple USB-Serial
      converter, it will appear under {\tt /dev/ttyUSB*}, where {\tt *}
      will be a number.
      In this example, we will use the {\tt miniterm.py} terminal
      application. PuTTy is another good alternative.

\begin{lstlisting}[style=block, language=bash]
> miniterm.py /dev/ttyUSB0 12800 # Open the terminal connection
\end{lstlisting}

      Note the Baud rate set to $12800$.

\item Initially the terminal will display nothing.
      Press the {\em reset} button on the Sasebo FPGA board labelled
      {\tt SW4} to reset the device.
      Note that there are two reset buttons, the other is labelled
      {\tt SW2}, and will reset the entire FPGA (rather than the
      scarv-soc). If you do this by accident, you will need to re-program
      the FPGA with the bitstream.

\item After pressing {\tt SW4}, if everything is working, you will see
      ``{\tt scarv-soc fsbl\textbackslash n}'' printed to the terminal.
      This indicates that everything is working as expected.

\begin{lstlisting}[style=block,language=bash]
> miniterm.py /dev/ttyUSB0 128000
--- Miniterm on /dev/ttyUSB0  128000,8,N,1 ---
--- Quit: Ctrl+] | Menu: Ctrl+T | Help: Ctrl+T followed by Ctrl+H ---
scarv-soc fsbl
\end{lstlisting}

\item Close miniterm (by pressing {\tt CTRL+]}).

\item We can upload any program to the FPGA using the script
    in \SOCDIR{bin/upload-program.py}.

\begin{lstlisting}[style=block,language=bash]
> ./bin/upload-program.py --help
usage: upload-program.py [-h] [--no-waiting] [--baud BAUD] [--goto GOTO]
                         port binary start

positional arguments:
  port          UART port to program over
  binary        The binary program to download
  start         Start address of the program as hex i.e. 0x20000000

optional arguments:
  -h, --help    show this help message and exit
  --no-waiting  Don't wait for the FSBL ready string, just send everything.
  --baud BAUD   UART port baud speed
  --goto GOTO   For telling the acquisition framework to re-enter the FSBL.
\end{lstlisting}

    This program will read continually from the specified UART port
    until it sees the ``{\tt scarv-soc fsbl\textbackslash n}'' message
    printed.  It then uploads the supplied program, and the CPU will
    automatically start executing it.

\item An example of this flow can be run using the make command:

\begin{lstlisting}[style=block,language=bash]
> make xilinx-interract
----------[Uploading Program]--------------
- Binary:  /home/work/scarv/scarv-soc-eo/work/examples/uart/uart.bin
- Address: 0x20000000
- Port:    /dev/ttyUSB0 & 128000
/home/work/scarv/scarv-soc-eo/bin/upload-program.py \
        --baud 128000 \
        /dev/ttyUSB0 \
        /home/work/scarv/scarv-soc-eo/work/examples/uart/uart.bin \
        0x20000000
Waiting for FSBL, please reset the target...
Programming /home/work/scarv/scarv-soc-eo/work/examples/uart/uart.bin, size=356, start=0x20000000
Programming: Done
-----------------[Done]--------------------
miniterm.py /dev/ttyUSB0 128000
--- Miniterm on /dev/ttyUSB0  128000,8,N,1 ---
--- Quit: Ctrl+] | Menu: Ctrl+T | Help: Ctrl+T followed by Ctrl+H ---
UART Echo Example Program
hello world
\end{lstlisting}
    
    This example builds the UART example program, uploads it to the
    FPGA and opens a terminal to talk to the program.
    Press any key, and the program running on the CPU will echo anything
    you type. Typing ``{\tt !}'' will quit the program.
\end{enumerate}

