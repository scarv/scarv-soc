
Here we describe the flow for re-creating the Xilinx project used to generate
a bitstream for the complete SoC.
The files for this flow live in {\tt \$SOC\_HOME/flow/xilinx}:

\begin{itemize}
\item {\tt scarv-soc-project.tcl} - The Vivado `tcl` script which creates 
    the project.

\item {\tt tb\_top\_behave.wcfg} - Simulation wave view configuration.

\item {\tt constraints-xc7k.xdc} - Pin constraints for targeting the
    SASEBO-GIII FPGA board.

\item {\tt scarv-soc-xc7k.bit} - A pre-generated bitfile.
\end{itemize}

\noindent Other relevent files are:

\begin{itemize}
\item {\tt bin/create-xilinx-xc7k-project.sh} - A bash script to automate the
    project re-creation steps.

\item {\tt rtl/xilinx/*} - Xilinx project specific RTL files.

\item {\tt verif/xilinx/*} - Testbench files for running simulations inside
Vivado.
\end{itemize}


\noindent Run the following commands to re-create the project:

\begin{itemize}

\item Make sure that the SOC environment is setup:

\begin{lstlisting}[style=block, language=bash]
> source ./bin/conf.sh
\end{lstlisting}

\item Make sure that the {\tt VIVADO\_ROOT} variable points to a Vivado
      installation:

\begin{lstlisting}[style=block, language=bash]
> export VIVADO\_ROOT=<path to installation>
\end{lstlisting}

    Note that the project was initially created using Vivado 2019.1.

\item Source the Vivado environment setup scripts:

\begin{lstlisting}[style=block, language=bash]
> source \$VIVADO\_ROOT/settings64.sh
\end{lstlisting}

\item Run the project re-creation scripts:

\begin{lstlisting}[style=block, language=bash]
> cd \$SOC\_HOME
> ./bin/create-xilinx-xc7k-project.sh
\end{lstlisting}

\item You can now open Vivado and the project file in
  {\tt \$SOC\_WORK/vivado/scarv-soc-xc7k/scarv-soc-xc7k.xpr}
  and start working with the project.

  There is also a pre-generated bitfile in 
  {\tt \$SOC\_HOME/flow/xilinx/}.

\end{itemize}

