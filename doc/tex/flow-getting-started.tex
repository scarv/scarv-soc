
\subsubsection{Dependencies}

The project uses the following Free and Open Source tools for
the hardware and software flows.
Most of these tools are pre-packaged for linux distributions.
All of our development is done using Ubuntu Linux 18.04 LTS.

\begin{itemize}[noitemsep]
\item \href{http://www.clifford.at/yosys/}{Yosys}
\item \href{https://symbiyosys.readthedocs.io/en/latest/}{Symbiyosys}
\item \href{https://boolector.github.io/}{Boolector}
\item \href{https://www.veripool.org/wiki/verilator}{Verilator}
\item \href{http://gtkwave.sourceforge.net/}{GTKWave}
\item \href{https://github.com/scarv/xcrypto.git}{SCARV Customised RISC-V toolchain}
\item \href{https://www.python.org/downloads/}{Python 3}
\item \href{http://www.doxygen.nl}{Doxygen}
\end{itemize}

In order to implement the project on a Xilinx FPGA, you will
need a copy of Xilinx Vivado 2019.2
\footnote{https://www.xilinx.com/products/design-tools/vivado.html}.
A completely open source tool flow is under development.

\subsubsection{Project Setup}

The following commands show how to checkout the \SCARVSOC project
repository, configure the project workspace and start running
builds and simulations.

\begin{enumerate}

\item Checkout the repository:

\begin{lstlisting}[language=bash,style=block]
> git clone git@github.com:scarv/scarv-soc.git scarv-soc
> cd ./scarv-soc
\end{lstlisting}

\item Checkout all of the repository submodules:

\begin{lstlisting}[language=bash,style=block]
> git submodule update --init --recursive
\end{lstlisting}

\item Set the path to your Yosys and Verilator installations:

\begin{lstlisting}[language=bash,style=block]
> export YOSYS_ROOT=<path to yosys installation>
> export VERILATOR_ROOT=<path to verilator installation>
\end{lstlisting}

\item Run the project workspace configuration script:

\begin{lstlisting}[language=bash,style=block]
> source ./bin/conf.sh
------------------------ SCARV SoC Project Setup ----------------------
SOC_HOME       = /home/ben/scarv/repos/scarv-soc
SOC_WORK       = /home/ben/scarv/repos/scarv-soc/work
VERILATOR_ROOT = /home/ben/tools/verilator
YOSYS_ROOT     = /home/ben/yosys
-----------------------------------------------------------------------
\end{lstlisting}

The exact output will differ based on your directory paths.

\end{enumerate}

After this point, you can start running commands using the project
make flow.
Note the \SOCHOME and \SOCWORK variables.
These are used extensively across the build system.
The first points to the root directory of the project.
The second points to where all build artifacts should be dumped.
By default, this is simple \SOCDIR{work}.

